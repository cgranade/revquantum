% \iffalse meta-comment
%
% Copyright (C) 2015 by Christopher Granade <cgranade@cgranade.com>
% ---------------------------------------------------------------------------
% This work may be distributed and/or modified under the
% conditions of the LaTeX Project Public License, either version 1.3
% of this license or (at your option) any later version.
% The latest version of this license is in
%   http://www.latex-project.org/lppl.txt
% and version 1.3 or later is part of all distributions of LaTeX
% version 2005/12/01 or later.
%
% This work has the LPPL maintenance status `maintained'.
%
% The Current Maintainer of this work is Christopher Granade (cgranade@cgranade.com).
%
% This work consists of the files revquantum.dtx and revquantum.ins
% and the derived filebase revquantum.sty.
%
% \fi
%
% \iffalse
%<*driver>
\ProvidesFile{revquantum.dtx}
%</driver>
%<package>\NeedsTeXFormat{LaTeX2e}[1999/12/01]
%<package>\ProvidesPackage{revquantum}
%<*package>
    [2015/11/04 v0.02 Hacks to make writing quantum papers for revtex4-1 less painful.]
%</package>
%
%<*driver>
\documentclass{ltxdoc}
\usepackage[pretty]{revquantum}[2015/11/04]
\usepackage{amsmath}
\EnableCrossrefs
\CodelineIndex
\RecordChanges
\begin{document}
  \DocInput{revquantum.dtx}
  \PrintChanges
  \PrintIndex
\end{document}
%</driver>
% \fi
%
% \CheckSum{0}
%
% \CharacterTable
%  {Upper-case    \A\B\C\D\E\F\G\H\I\J\K\L\M\N\O\P\Q\R\S\T\U\V\W\X\Y\Z
%   Lower-case    \a\b\c\d\e\f\g\h\i\j\k\l\m\n\o\p\q\r\s\t\u\v\w\x\y\z
%   Digits        \0\1\2\3\4\5\6\7\8\9
%   Exclamation   \!     Double quote  \"     Hash (number) \#
%   Dollar        \$     Percent       \%     Ampersand     \&
%   Acute accent  \'     Left paren    \(     Right paren   \)
%   Asterisk      \*     Plus          \+     Comma         \,
%   Minus         \-     Point         \.     Solidus       \/
%   Colon         \:     Semicolon     \;     Less than     \<
%   Equals        \=     Greater than  \>     Question mark \?
%   Commercial at \@     Left bracket  \[     Backslash     \\
%   Right bracket \]     Circumflex    \^     Underscore    \_
%   Grave accent  \`     Left brace    \{     Vertical bar  \|
%   Right brace   \}     Tilde         \~}
%
%
% \changes{v0.02}{2015/11/04}{Initial public version.}
%
% \DoNotIndex{\newcommand,\newenvironment}
%
% \providecommand*{\url}{\texttt}
% \GetFileInfo{revquantum.dtx}
% \title{The \textsf{revquantum} package}
% \author{Christopher Granade \\ \url{cgranade@cgranade.com}}
% \date{\fileversion~from \filedate}
%
% \maketitle
%
% \section{Introduction}
%
% The \textsf{revquantum} package provides a number of useful hacks to solve
% common annoyances with the \textsf{revtex4-1} package, and to define notation
% in common use within quantum information.
%
% \section{Usage}
%
% \textbf{NB:} \textsf{revquantum} must be loaded first unless |nobibtexhacks|
% is passed.
%
%
% \DescribeMacro{\booloption}
% \DescribeMacro{\newnew}
%
% These commands are used internally by \textsf{revquantum} to define
% new boolean package options, and to declare new kinds of |\newcommand|
% commands. For example, |\newnew| is used to define |\newoperator|,
% which in turn defines new commands for named mathematical operators (e.g. $\Tr$).
%
%
% \DescribeMacro{\sectionautorefname}
% \DescribeMacro{\algorithmautorefname}
% \DescribeMacro{\equationautorefname}
% \DescribeMacro{\lemmaautorefname}
%
% These commands configure \textsf{hyperref}'s \texttt{autoref} command
% for use with \textsf{revtex4-1}, so that \texttt{autoref} correctly
% describes Section, Algorithm, and Lemma, and also follows the standard
% notation for equation references.
%
% \DescribeMacro{\TODO}
%
%
% \DescribeMacro{\todolist}
%
%
%
%
% \DescribeMacro{\ii}
%
%
% \DescribeMacro{\dd}
%
%
% \DescribeMacro{\defeq}
%
%
% \DescribeMacro{\expect}
%
%
% \DescribeMacro{\id}
%
%
% \DescribeMacro{\newaffil}
%

%
% \StopEventually{}
%
% \section{Implementation}
%
% \iffalse
%<*package>
% \fi
%

% \subsection{Options Handling} %%%%%%%%%%%%%%%%%%%%%%%%%%%%%%%%%%%%%%%%%%%%%%

% We will need to define a few options to make the package nicer to use. We do
% so by making a new command, |\booloption| \marg{boolname} \marg{optionname}
% \marg{default}.

%    \begin{macrocode}

\usepackage{ifthen}

%    \end{macrocode}
%
% \begin{macro}{\booloption}
%    \begin{macrocode}
\newcommand{\booloption}[3]{
    \newboolean{#1}
    \setboolean{#1}{#3}
    \ifthenelse{\equal{#3}{true}}{
%    \end{macrocode}
% Default is true, so we need a "no" option to
% turn off the new boolean.
%    \begin{macrocode}
        \DeclareOption{no#2}{\setboolean{#1}{false}}
    }{
%    \end{macrocode}
% Default is false, so we need an option to turn
% on the new boolean.
%    \begin{macrocode}
        \DeclareOption{#2}{\setboolean{#1}{true}}
    }
}
%    \end{macrocode}
% \end{macro}

% We then use this new command to define the options for \textsf{revquantum},
% |final|, |pretty|, |uselistings| and |nobibtexhacks|.

%    \begin{macrocode}

\booloption{@final}{final}{false}
\booloption{@pretty}{pretty}{false}
\booloption{@uselistings}{uselistings}{false}
\booloption{@bibtexhacks}{bibtexhacks}{true}

\ProcessOptions\relax

%    \end{macrocode}

% \subsection{Unforgivable BibTeX Hacks} %%%%%%%%%%%%%%%%%%%%%%%%%%%%%%%%%%%%%

% These hacks include the title of each reference in the BibTeX output by
% redefining the part of \textsf{revtex4-1} on the fly which is responsible
% for writing out the bibdata.

%    \begin{macrocode}

\ifthenelse{\boolean{@bibtexhacks}}{\def\@bibdataout@aps{%
 \immediate\write\@bibdataout{%
  @CONTROL{%
   apsrev41Control,author="08",editor="1",pages="0",title="0",year="1",eprint="1"%
  }%
 }%
 \if@filesw
  \immediate\write\@auxout{\string\citation{apsrev41Control}}%
 \fi
}}{}

%    \end{macrocode}

% \subsection{Imports} %%%%%%%%%%%%%%%%%%%%%%%%%%%%%%%%%%%%%%%%%%%%%%%%%%%%%%%

% Here, we import a few other useful packages and configure them according to
% the options passed by the user.

%    \begin{macrocode}

\RequirePackage[breaklinks=true]{hyperref}
\RequirePackage[usenames,dvipsnames]{color}
\RequirePackage{braket}
\RequirePackage{graphicx}
\RequirePackage[english]{babel}
\ifthenelse{\boolean{@pretty}}{
    \RequirePackage{mathpazo}
}{}
\ifthenelse{\boolean{@uselistings}}{
    \RequirePackage{listings}
    \RequirePackage{textcomp} % Make sure we have a ` for writing
%    \end{macrocode}
% Mathematica.
%    \begin{macrocode}
}{}
%    \end{macrocode}

% \subsubsection{\textsf{hyperref} Setup}

% Get rid of \textsf{hyperref}'s ugly boxes.
%From:http://tex.stackexchange.com/a/51349
%    \begin{macrocode}
\hypersetup{
  colorlinks   = true, %Colours links instead of ugly boxes
  urlcolor     = blue, %Colour for external hyperlinks
  linkcolor    = blue, %Colour of internal links
  citecolor    = red %Colour of citations
}

%    \end{macrocode}

% \subsubsection{\textsf{autoref} Setup}

% \begin{macro}{\sectionautorefname}
%    \begin{macrocode}
\def\sectionautorefname{Section}
%    \end{macrocode}
% \end{macro}
%
%
% \begin{macro}{\algorithmautorefname}
%    \begin{macrocode}
\def\algorithmautorefname{Algorithm}
%    \end{macrocode}
% \end{macro}
%
%
% \begin{macro}{\equationautorefname}
% See http://tex.stackexchange.com/a/66150.
%    \begin{macrocode}
\def\equationautorefname~#1\null{(#1)\null}
%    \end{macrocode}
% \end{macro}
%
%
% \begin{macro}{\lemmaautorefname}
%    \begin{macrocode}
\newcommand{\lemmaautorefname}{Lemma}
%    \end{macrocode}
% \end{macro}
%
%    \begin{macrocode}

%    \end{macrocode}

% \subsubsection{Theorem Environments}

%    \begin{macrocode}

\newtheorem{theorem}{Theorem}
\newtheorem{lemma}{Lemma}

%    \end{macrocode}

% \subsubsection{\textsf{algorithm} and \textsf{algpseudocode} Setup}

%    \begin{macrocode}

\RequirePackage{algorithm}
\RequirePackage{algpseudocode}
    \renewcommand{\algorithmicrequire}{\textbf{Input:}}
    \renewcommand{\algorithmicensure}{\textbf{Output:}}
    \newcommand{\inlinecomment}[1]{\Comment {\footnotesize #1} \normalsize}
    \newcommand{\linecomment}[1]{\State \(\triangleright\) {\footnotesize #1} \normalsize}

%    \end{macrocode}

% \subsubsection{\textsf{listings} Setup}

%    \begin{macrocode}

\ifthenelse{\boolean{@uselistings}}{
    \definecolor{comment-color}{gray}{0.5}

    \lstset{
        basicstyle=\footnotesize,
        commentstyle=\color{comment-color},
        frame=lines,
        gobble=4,
        numbers=left,
        numberstyle=\tiny, stepnumber=5,
        numbersep=5pt
    }

    \lstdefinestyle{matlab}{
        language=MATLAB,
        mathescape=true
    }

    \lstdefinestyle{python}{
        language=Python,
        mathescape=true,
        showstringspaces=false,
    }

    \lstdefinestyle{mathematica}{
        language=Mathematica,
        upquote=true, % Needed to deal with the context symbol `.'
        literate=
            {->}{$\to$}1
            {!=}{$\neq$}1
            {\\[DoubleStruckOne]}{${\id}$}1
            {\\[Sigma]}{$\sigma$}1
            {(x)}{$\otimes$}1 % CG: I have the distinct impression this will break. Badly.
    }
}{}

%    \end{macrocode}

% \subsection{TODO Support} %%%%%%%%%%%%%%%%%%%%%%%%%%%%%%%%%%%%%%%%%%%%%%%%%%

% These commands provide a way of marking items as needing to be done before
% the final version (denoted by the |final| package option).

% \begin{macro}{\todo}
%    \begin{macrocode}

\ifthenelse{\boolean{@final}}{
    \newcommand{\todo}[1]{
        \PackageError{revquantum}{Unaddressed TODO}
        \rq@todo{#1}
    }
}{
    \newcommand{\todo}[1]{
        \PackageWarning{revquantum}{Unaddressed TODO}
        \rq@todo{#1}
    }
}

%    \end{macrocode}
% \end{macro}
%
% \begin{macro}{\rq@todo}
%    \begin{macrocode}
\newcommand{\rq@todo}[1]{
    {\color{Plum} #1}
}
%    \end{macrocode}
% \end{macro}
%
%
% \begin{macro}{\TODO}
%    \begin{macrocode}
\newcommand{\TODO}{\todo{TODO}}
%    \end{macrocode}
% \end{macro}
%
%
% \begin{macro}{\todolist}
%    \begin{macrocode}
\newcommand{\todolist}[1]{\todo{
    \begin{itemize}
        #1
    \end{itemize}
}}

%    \end{macrocode}
% \end{macro}


% \subsection{Notation}
%
%
% \begin{macro}{\newnew}
%    \begin{macrocode}
\newcommand{\newnew}[2]{
    \expandafter\newcommand\csname new#1\endcsname[1]{
      \expandafter\newcommand\csname ##1\endcsname{#2{##1}}
    }
}
%    \end{macrocode}
% \end{macro}
%
%    \begin{macrocode}
\newnew{operator}{\operatorname}
\newnew{rm}{\mathrm}

\newoperator{Tr}
\newoperator{Cov}
\newoperator{supp}
\newoperator{diag}
\newoperator{rank}

%    \end{macrocode}
%
% \begin{macro}{\ii}
%    \begin{macrocode}
\newcommand{\ii}{\mathrm{i}} % Outside what newnew currently supports.
%    \end{macrocode}
% \end{macro}
%
%
% \begin{macro}{\dd}
%    \begin{macrocode}
\newcommand{\dd}{\mathrm{d}} % Outside what newnew currently supports.
%    \end{macrocode}
% \end{macro}
%
%    \begin{macrocode}
\newrm{e}

\newrm{pa}
\newrm{TVD}
\newrm{T}

%    \end{macrocode}
%
% \begin{macro}{\defeq}
%    \begin{macrocode}
\newcommand{\defeq}{\mathrel{:=}}
%    \end{macrocode}
% \end{macro}
%
%
% \begin{macro}{\expect}
%    \begin{macrocode}
\newcommand{\expect}{\mathbb{E}}
%    \end{macrocode}
% \end{macro}
%
%
% \begin{macro}{\id}
%    \begin{macrocode}
\newcommand{\id}{\openone}
%    \end{macrocode}
% \end{macro}
%
%    \begin{macrocode}

%    \end{macrocode}

% \subsection{Affiliation Database}
%
% \begin{macro}{\newaffil}
%    \begin{macrocode}
\newcommand{\newaffil}[2]{
    \expandafter\newcommand\csname affil#1\endcsname{
        \affiliation{
            #2
        }
    }
}
%    \end{macrocode}
% \end{macro}
%
%    \begin{macrocode}

\newaffil{TODO}{\TODO}

\newaffil{EQUS}{
    Centre for Engineered Quantum Systems,
    University of Sydney,
    Sydney, NSW, Australia
}
\newaffil{USydPhys}{
    School of Physics,
    University of Sydney,
    Sydney, NSW, Australia
}
\newaffil{IQC}{
    Institute for Quantum Computing,
    University of Waterloo,
    Waterloo, ON, Canada
}
\newaffil{UWAMath}{
    Department of Applied Mathematics,
    University of Waterloo,
    Waterloo, ON, Canada
}
\newaffil{UWChem}{
    Department of Chemistry,
    University of Waterloo,
    Waterloo, ON, Canada
}
\newaffil{PI}{
    Perimeter Institute for Theoretical Physics,
    31 Caroline St. N,
    Waterloo, Ontario, Canada N2L 2Y5
}
\newaffil{CIFAR}{
    Canadian Institute for Advanced Research,
    Toronto, ON, Canada
}
\newaffil{IBMTJW}{
    IBM T. J. Watson Research Center,
    Yorktown Heights, New York 10598, USA
}

\endinput%    \end{macrocode}

%
% \iffalse
%</package>
% \fi
%
% \Finale
\endinput
